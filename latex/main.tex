\documentclass[
% -- opções da classe memoir --
12pt,				% tamanho da fonte
openright,			% capítulos começam em pág ímpar
oneside,			% para impressão em pagina unica
a4paper,			% tamanho do papel. 
english,			% idioma adicional para hifenização
french,				% idioma adicional para hifenização
spanish,			% idioma adicional para hifenização
brazil,				% o último idioma é o principal do documento
]{abntex2}

\usepackage{lmodern}			% Usa a fonte Latin Modern
\usepackage[T1]{fontenc}		% Selecao de codigos de fonte.
\usepackage[utf8]{inputenc}		% acento
\usepackage{indentfirst}		% Indenta o primeiro parágrafo
\usepackage{color}				% Controle das cores
\usepackage{graphicx}			% Inclusão de gráficos
\usepackage{microtype} 			% para melhorias de justificação
\usepackage{placeins}
\usepackage{bigfoot}            % verbatim
\usepackage[numbered,framed]{matlab-prettifier}
\usepackage{multicol}
\usepackage{multirow}
\usepackage{lipsum}				% para geração de dummy text
\usepackage[brazilian,hyperpageref]{backref} % citações na bibl
\usepackage[alf]{abntex2cite}	% Citações padrão ABNT

% Configurações do pacote backref
% Usado sem a opção hyperpageref de backref
\renewcommand{\backrefpagesname}{Citado na(s) página(s):~}
% Texto padrão antes do número das páginas
\renewcommand{\backref}{}
% Define os textos da citação
\renewcommand*{\backrefalt}[4]{
	\ifcase #1 %
	Nenhuma citação no texto.%
	\or
	Citado na página #2.%
	\else
	Citado #1 vezes nas páginas #2.%
	\fi}%
	
	
% ----------------------------------------------------------------

%redefine a capa
\renewcommand{\imprimircapa}{%
	\begin{capa}%
		\center
		\ABNTEXchapterfont\Large \textbf{UNIVERSIDADE FEDERAL DE SERGIPE}
		\\
		\vspace*{1cm}
		{\ABNTEXchapterfont\large\imprimirautor}
		\vfill
		\begin{center}
			\ABNTEXchapterfont\bfseries\LARGE\imprimirtitulo
		\end{center}
		\vfill
		\large\imprimirlocal \\
		\large\imprimirdata
		\vspace*{1cm}
	\end{capa}
}
% ---
%----
% Informações de dados para CAPA e FOLHA DE ROSTO
% ---
\titulo{Encriptação AES}
\autor{Iuri Rodrigo Ferreira Alves da silva\\Gregory Medeiros Melgaço Pereira\\Raul Rodrigo Silva de Andrade \\ Rafael Castro Nunes \\ Ruan Robert Bispo dos Santos \\ Vítor do Bomfim Almeida Carvalho}
\local{São Cristóvão,SE}
\data{\today}
\instituicao{%
	Universidade Federal De Sergipe
	\par
	Faculdade de Engenharia Eletrônica
	\par
	Redes e Comunicações}
\tipotrabalho{Relatório técnico}
% O preambulo deve conter o tipo do trabalho, o objetivo, 
% o nome da instituição e a área de concentração 
\preambulo{Relatório em conformidade com as normas ABNT}
% ---

% ---
% Configurações de aparência do PDF final

% alterando o aspecto da cor azul
\definecolor{blue}{RGB}{41,5,195}

% informações do PDF
\makeatletter
\hypersetup{
	%pagebackref=true,
	pdftitle={\@title}, 
	pdfauthor={\@author},
	pdfsubject={\imprimirpreambulo},
	pdfcreator={LaTeX with abnTeX2},
	pdfkeywords={abnt}{latex}{abntex}{abntex2}{relatório técnico}, 
	colorlinks=true,       		% false: boxed links; true: colored links
	linkcolor=blue,          	% color of internal links
	citecolor=blue,        		% color of links to bibliography
	filecolor=magenta,      		% color of file links
	urlcolor=blue,
	bookmarksdepth=4
}
\makeatother

% --- 
% O tamanho do parágrafo é dado por:
\setlength{\parindent}{1.3cm}
\setlength{\parskip}{0.2cm}  

% compila o indice
% ---
\makeindex

%----------------------------------------------------------------

% Início do documento
% ----
\begin{document}

	\lstset{language=Matlab} 
	\selectlanguage{brazil}
	\frenchspacing  %retira espaço obsoleto entre frase
	
	% Capa
	\imprimircapa
	
	% (o * indica que haverá a ficha bibliográfica)
	\imprimirfolhaderosto*
	
	% RESUMO
	\include{resumo}
	
	%lista de ilustrações
	\pdfbookmark[0]{\listfigurename}{lof}
	\listoffigures*
	\pagebreak
	% ---

	%lista de tabelas
	\pdfbookmark[0]{\listtablename}{lot}
	\listoftables*
	\pagebreak
	% ---
	
	% sumario
	\pdfbookmark[0]{\contentsname}{toc}
	\tableofcontents*
	\pagebreak
	
	%%%%%%%%%%%%%%%%%%%%%%%%%%%%%%%%%%%%%%%%%
	%%%%%%%%% partes
	%%%%%%%%%%%%%%%%%%%%%%%%%%%%%%%%%%%%%%%%%
	
	\include{intro}
	\phantompart      %aparecer no indice
	
	\chapter{Revisão bibliográfica}

\section{AES}

Com o avanço tecnológico alcançado pelo homem, aumentou-se a necessidade por segurança em todos os aspectos. As trocas de informações se tornaram mais intensas e se viu necessária a presença de ferramentas para a proteção de arquivos armazenados em bancos de dados. A criptografia é usada como um desses métodos para assegurar sigilo de qualquer tipo de informação virtual, tornando-as códigos a serem decriptadas apenas pelo receptor da mensagem. 

Em \cite{lu2002integrated} é proposto um método para integrar a criptografia e descriptografia AES em uma ferramenta funcional. Para implementar esse algoritmo é mostrado que são necessárias 5 operações principais: AddRoundKey, SubBytes, ShiftRows, MixColumns e keyExpansion, em que são detalhadas cada uma dessas operações para implementação do algoritmo. Ao término do trabalho, conclui-se que a criptografia AES mostra-se bastante eficiente e de baixa complexidade, proporcionando altas taxas de processamento em hardwares tanto no processo de encriptação como na decriptação.

Em \cite{oliveira2012criptografia} é ressaltado que as técnicas computacionais tornaram-se fundamentais para que os requisitos da proteção a informação sejam atendidos, apresentando neste cenário dois tipos básicos para criptografia: simétrica e assimétrica.
Na criptografia simétrica, que se encontra o AES, existe a presença de uma chave secreta, usada tanto pelo remetente quanto o destinatário e a ideia é que se use um algoritmo de deciframento junto com esta chave para cifrar e decifrar a mensagem. As principais vantagens observadas são a simplicidade e rapidez para executar os processos criptográficos.

Na criptografia assimétrica, diferentemente da simétrica, existem duas chaves, uma secreta e outra pública. A chave pública pode ficar disponível para qualquer pessoa que queira se comunicar de modo seguro, mas a secreta é de posse apenas de cada titular e é com ela que o destinatário poderá decodificar uma mensagem que foi criptografada para ele com sua respectiva chave pública. A vantagem desse sistema é permitir a qualquer um enviar uma mensagem secreta, apenas utilizando a chave pública de quem irá recebê-la, a desvantagem é a complexidade empregada no desenvolvimento dos algoritmos que devem ser capazes de reconhecer a dupla de chaves existentes.

já no artigo \cite{ribeiro2001estudo} é feito uma comparação entre DES(Data encryption Standard) e AES(Advanced Encryption Standard), onde é indagado, no começo, quais as melhorias trazidas pelo AES, já que este substituiu o DES como sistema de criptografia. Sendo que as principais diferenças entre o DES e o AES são os algoritmos concorrentes, onde no DES só tinha 1, no AES tiveram vários.

Para análise de algoritmos, ambos foram testados na mesma máquina para obtenção dos resultados. No experimento foi obtido que entre o DES e o Rijndael o tempo do DES é menor, junto ao tamanho do arquivo criado.

	\phantompart
	
	\chapter{Objetivos}
	\phantompart
	
	\include{funda}
	\phantompart
	
	\chapter{Formulação do problema}
	\phantompart
	
	\chapter{Resultados Obtidos}
	\phantompart
	
	\chapter{Conclusão}
	\phantompart
	
	
	%%%%%%%%%%%%%%%%%%%%%%%%%%%%%%%%%%%%%%%%%
	%Referência
	%%%%%%%%%%%%%%%%%%%%%%%%%%%%%%%%%%%%%%%%%
    \nocite{harrison2007aes,lu2002integrated,hamalainen2006design,oliveira2012criptografia,coutinho,stallings,burnett,dacriptografia,ribeiro2001estudo}
    
	\bibliography{biblio}
	
	
	%%%%%%%%%%%%%%%%%%%%%%%%%%%%%%%%%%%%%%%%%
	%the end
	%%%%%%%%%%%%%%%%%%%%%%%%%%%%%%%%%%%%%%%%%
\end{document}
